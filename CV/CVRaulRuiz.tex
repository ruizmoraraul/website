% -- Encoding UTF-8 without BOM
% -- XeLaTeX => PDF (BIBER)

\documentclass[]{cv-style}          % Add 'print' as an option into the square bracket to remove colours from this template for printing. 
                                    % Add 'espanol' as an option into the square bracket to change the date format of the Last Updated Text

\usepackage{hyperref}
\usepackage{fontawesome}

\sethyphenation[variant=british]{english}{} % Add words between the {} to avoid them to be cut 

\begin{document}

\header{Raúl}{Ruiz}           % Your name
\lastupdated

%----------------------------------------------------------------------------------------
%	SIDEBAR SECTION  -- In the aside, each new line forces a line break
%----------------------------------------------------------------------------------------

\begin{aside}
%
\section{Personal Profile}
Raúl Ruiz Mora
C/ Alcúdia de Crespins,
12, pta 26, 46019
Valencia, Spain
~
\href{mailto:ruizmoraraul@gmail.com}{\small ruizmoraraul@gmail.com}
~
(+34) 658 488 592
\end{aside}

\vspace{1cm}

I'm a hardworking person interested in abstract algebra and logic. I consider myself a helpful and gentle person and someone with whom is easy to wok with. I try to do my best when I'm studiyng or working.

\section{Education}

\begin{entrylist}
\entry
    {2021--Now}
    {MSc in Logic}
    {Amsterdam, Netherlands}
    {\jobtitle{Univeristy of Amsterdam. Institute of Logic Language and Computation (ILLC)}\\
    Average grade: \bf --/10. ECTS: 120}

\entry
    {2017--2021}
    {BSc degree in Mathematics}
    {Valencia, Spain}
    {\jobtitle{University of Valencia}. Extraordinary Prize of Degree.\\
    Average grade: {\bf 9.44}/10. ECTS:240 (135 passed with honours)}
\end{entrylist}

\section{Undergraduate work}

\begin{entrylist}
\entry
    {2021}
    {''A comic page for the first isomorphism theorem"}
    {Journal of Mathematics and the Arts}
    {\jobtitle{E. Cosme Llópez, R. Ruiz
Mora, N. Tamarit}, Submitted, Under revision.}
\entry
    {2021}
    {BSc thesis: ''Els Teoremes d'Isomorfisme en àlgebres heterogènies"}
    {}
    {\jobtitle{Universitat de València, Facultat de Ciències Matemàtiques}\\''Isomorphism theorems in many-sorted algebras". Interested in the abstraction of algebraic structures and logic, I did my Bacholr's thesis in the theory of universal many-sorted algebras, finally proving the isomorphism theorems. Work under the supervision of \hyperlink{https://www.uv.es/coslloen/}{Enric Cosme LLópez}.}
\entry
    {2020-2021}
    {Course in Universal Algebra (as lecturer)}
    {120h.}
    {\jobtitle{Universitat de València, Facultat de Ciències Matemàtiques}\\
    Practicum at the area of algebra from the mathematics department of the University of Valencia. The aim was to introduce universal algebra until giving a proof for the Birkhoff theorem of varieties and theorems about Mal'cev conditions. The course conosisted in 18 lectures of 1 hour and the whole practicum were in 120 hours in total.}
\end{entrylist}

\section{Projects}

\begin{entrylist}
\entry
    {2019--2021}
    {AdR Facultat de Matemàtiques, Junta de Centre \& CAT}
    {}
    {\jobtitle{Universitat de valència, Facultat de Ciències Matemàtiques}\\
    During two school years I was elected as representative of my course. Some of my occupations were to manage the online teaching experience in the context of the pandemic and the lockdown, and to represent my classmates in two of the most important councils of the Faculty of Maths: Junta de centre de la facultat de matemàtiques (Centre Board of the faculty of maths) and CAT (Academic Committee for Qualifications).}
\entry
    {2019--2021}
    {PEC - Programa d'Estudiants Correctors}
    {}
    {\jobtitle{Universitat de València, Facultat de Ciències Matemàtiques} \\
    Senior students assessment program. Universitat de València.\\
    The aim of the project is to help first-year students to write with mathematic and logic language in a proper way. I was elected because of my logical rigor when writing matemathics.}
\end{entrylist}

\newpage
\section{Languages}

\begin{entrylist}
\entry
    {Spanish}
    {Mothertongue}
    {}
    {}
\entry
    {Catalan}
    {Mothertongue}
    {}
    {I have done all my education in catalan. Moreover, in October, 2017 I got the certificate of C1 by the "Junta Qualificadora dels Coneixements del Valencia" (Valencian Knowledge Qualifying Council).}
\entry
    {English}
    {C1}
    {}
    {I got the CAE Cambridge certificate in July, 2021 with a mark of 196.}
\end{entrylist}

\section{Courses}

\begin{entrylist}
\entry
    {Sep. 2019}
    {Course in project magement}
    {20 h.}
    {\jobtitle{La Nau de la Universitat de Valencia}\\
    Course about how to plan, manage and present a project.}
\entry
    {Sep. 2019}
    {Ofimatic presentations}
    {20 h.}
    {\jobtitle{La Nau de la Universitat de Valencia}\\
    Course about how to make an ofimatic presentation with "Powerpoint" and "canva".}
\entry
    {May 2017}
    {High academic performance math course}
    {4 h.}
    {\jobtitle{Universitat de València, Facultat de Ciències Matemàtiques}\\
    A course for high-school students with the best grades about to introduce them into the mathematical reasoning and proofs in mathematics.}
\end{entrylist}

\section{Software}

\begin{entrylist}
\entry
    {High level}
    {Matlab, Mathematica, R, C++, Maxima, Latex}
    {}
    {}
\entry
    {Basic level}
    {Microsoft office (Excel, Word, Powerpoint), Canva}
    {}
    {}
\end{entrylist}

\end{document}